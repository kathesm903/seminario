
    




    
\documentclass[11pt]{article}

    
    \usepackage[breakable]{tcolorbox}
    \tcbset{nobeforeafter} % prevents tcolorboxes being placing in paragraphs
    \usepackage{float}
    \floatplacement{figure}{H} % forces figures to be placed at the correct location
    
    \usepackage[T1]{fontenc}
    % Nicer default font (+ math font) than Computer Modern for most use cases
    \usepackage{mathpazo}

    % Basic figure setup, for now with no caption control since it's done
    % automatically by Pandoc (which extracts ![](path) syntax from Markdown).
    \usepackage{graphicx}
    % We will generate all images so they have a width \maxwidth. This means
    % that they will get their normal width if they fit onto the page, but
    % are scaled down if they would overflow the margins.
    \makeatletter
    \def\maxwidth{\ifdim\Gin@nat@width>\linewidth\linewidth
    \else\Gin@nat@width\fi}
    \makeatother
    \let\Oldincludegraphics\includegraphics
    % Set max figure width to be 80% of text width, for now hardcoded.
    \renewcommand{\includegraphics}[1]{\Oldincludegraphics[width=.8\maxwidth]{#1}}
    % Ensure that by default, figures have no caption (until we provide a
    % proper Figure object with a Caption API and a way to capture that
    % in the conversion process - todo).
    \usepackage{caption}
    \DeclareCaptionLabelFormat{nolabel}{}
    \captionsetup{labelformat=nolabel}

    \usepackage{adjustbox} % Used to constrain images to a maximum size 
    \usepackage{xcolor} % Allow colors to be defined
    \usepackage{enumerate} % Needed for markdown enumerations to work
    \usepackage{geometry} % Used to adjust the document margins
    \usepackage{amsmath} % Equations
    \usepackage{amssymb} % Equations
    \usepackage{textcomp} % defines textquotesingle
    % Hack from http://tex.stackexchange.com/a/47451/13684:
    \AtBeginDocument{%
        \def\PYZsq{\textquotesingle}% Upright quotes in Pygmentized code
    }
    \usepackage{upquote} % Upright quotes for verbatim code
    \usepackage{eurosym} % defines \euro
    \usepackage[mathletters]{ucs} % Extended unicode (utf-8) support
    \usepackage[utf8x]{inputenc} % Allow utf-8 characters in the tex document
    \usepackage{fancyvrb} % verbatim replacement that allows latex
    \usepackage{grffile} % extends the file name processing of package graphics 
                         % to support a larger range 
    % The hyperref package gives us a pdf with properly built
    % internal navigation ('pdf bookmarks' for the table of contents,
    % internal cross-reference links, web links for URLs, etc.)
    \usepackage{hyperref}
    \usepackage{longtable} % longtable support required by pandoc >1.10
    \usepackage{booktabs}  % table support for pandoc > 1.12.2
    \usepackage[inline]{enumitem} % IRkernel/repr support (it uses the enumerate* environment)
    \usepackage[normalem]{ulem} % ulem is needed to support strikethroughs (\sout)
                                % normalem makes italics be italics, not underlines
    \usepackage{mathrsfs}
    

    
    % Colors for the hyperref package
    \definecolor{urlcolor}{rgb}{0,.145,.698}
    \definecolor{linkcolor}{rgb}{.71,0.21,0.01}
    \definecolor{citecolor}{rgb}{.12,.54,.11}

    % ANSI colors
    \definecolor{ansi-black}{HTML}{3E424D}
    \definecolor{ansi-black-intense}{HTML}{282C36}
    \definecolor{ansi-red}{HTML}{E75C58}
    \definecolor{ansi-red-intense}{HTML}{B22B31}
    \definecolor{ansi-green}{HTML}{00A250}
    \definecolor{ansi-green-intense}{HTML}{007427}
    \definecolor{ansi-yellow}{HTML}{DDB62B}
    \definecolor{ansi-yellow-intense}{HTML}{B27D12}
    \definecolor{ansi-blue}{HTML}{208FFB}
    \definecolor{ansi-blue-intense}{HTML}{0065CA}
    \definecolor{ansi-magenta}{HTML}{D160C4}
    \definecolor{ansi-magenta-intense}{HTML}{A03196}
    \definecolor{ansi-cyan}{HTML}{60C6C8}
    \definecolor{ansi-cyan-intense}{HTML}{258F8F}
    \definecolor{ansi-white}{HTML}{C5C1B4}
    \definecolor{ansi-white-intense}{HTML}{A1A6B2}
    \definecolor{ansi-default-inverse-fg}{HTML}{FFFFFF}
    \definecolor{ansi-default-inverse-bg}{HTML}{000000}

    % commands and environments needed by pandoc snippets
    % extracted from the output of `pandoc -s`
    \providecommand{\tightlist}{%
      \setlength{\itemsep}{0pt}\setlength{\parskip}{0pt}}
    \DefineVerbatimEnvironment{Highlighting}{Verbatim}{commandchars=\\\{\}}
    % Add ',fontsize=\small' for more characters per line
    \newenvironment{Shaded}{}{}
    \newcommand{\KeywordTok}[1]{\textcolor[rgb]{0.00,0.44,0.13}{\textbf{{#1}}}}
    \newcommand{\DataTypeTok}[1]{\textcolor[rgb]{0.56,0.13,0.00}{{#1}}}
    \newcommand{\DecValTok}[1]{\textcolor[rgb]{0.25,0.63,0.44}{{#1}}}
    \newcommand{\BaseNTok}[1]{\textcolor[rgb]{0.25,0.63,0.44}{{#1}}}
    \newcommand{\FloatTok}[1]{\textcolor[rgb]{0.25,0.63,0.44}{{#1}}}
    \newcommand{\CharTok}[1]{\textcolor[rgb]{0.25,0.44,0.63}{{#1}}}
    \newcommand{\StringTok}[1]{\textcolor[rgb]{0.25,0.44,0.63}{{#1}}}
    \newcommand{\CommentTok}[1]{\textcolor[rgb]{0.38,0.63,0.69}{\textit{{#1}}}}
    \newcommand{\OtherTok}[1]{\textcolor[rgb]{0.00,0.44,0.13}{{#1}}}
    \newcommand{\AlertTok}[1]{\textcolor[rgb]{1.00,0.00,0.00}{\textbf{{#1}}}}
    \newcommand{\FunctionTok}[1]{\textcolor[rgb]{0.02,0.16,0.49}{{#1}}}
    \newcommand{\RegionMarkerTok}[1]{{#1}}
    \newcommand{\ErrorTok}[1]{\textcolor[rgb]{1.00,0.00,0.00}{\textbf{{#1}}}}
    \newcommand{\NormalTok}[1]{{#1}}
    
    % Additional commands for more recent versions of Pandoc
    \newcommand{\ConstantTok}[1]{\textcolor[rgb]{0.53,0.00,0.00}{{#1}}}
    \newcommand{\SpecialCharTok}[1]{\textcolor[rgb]{0.25,0.44,0.63}{{#1}}}
    \newcommand{\VerbatimStringTok}[1]{\textcolor[rgb]{0.25,0.44,0.63}{{#1}}}
    \newcommand{\SpecialStringTok}[1]{\textcolor[rgb]{0.73,0.40,0.53}{{#1}}}
    \newcommand{\ImportTok}[1]{{#1}}
    \newcommand{\DocumentationTok}[1]{\textcolor[rgb]{0.73,0.13,0.13}{\textit{{#1}}}}
    \newcommand{\AnnotationTok}[1]{\textcolor[rgb]{0.38,0.63,0.69}{\textbf{\textit{{#1}}}}}
    \newcommand{\CommentVarTok}[1]{\textcolor[rgb]{0.38,0.63,0.69}{\textbf{\textit{{#1}}}}}
    \newcommand{\VariableTok}[1]{\textcolor[rgb]{0.10,0.09,0.49}{{#1}}}
    \newcommand{\ControlFlowTok}[1]{\textcolor[rgb]{0.00,0.44,0.13}{\textbf{{#1}}}}
    \newcommand{\OperatorTok}[1]{\textcolor[rgb]{0.40,0.40,0.40}{{#1}}}
    \newcommand{\BuiltInTok}[1]{{#1}}
    \newcommand{\ExtensionTok}[1]{{#1}}
    \newcommand{\PreprocessorTok}[1]{\textcolor[rgb]{0.74,0.48,0.00}{{#1}}}
    \newcommand{\AttributeTok}[1]{\textcolor[rgb]{0.49,0.56,0.16}{{#1}}}
    \newcommand{\InformationTok}[1]{\textcolor[rgb]{0.38,0.63,0.69}{\textbf{\textit{{#1}}}}}
    \newcommand{\WarningTok}[1]{\textcolor[rgb]{0.38,0.63,0.69}{\textbf{\textit{{#1}}}}}
    
    
    % Define a nice break command that doesn't care if a line doesn't already
    % exist.
    \def\br{\hspace*{\fill} \\* }
    % Math Jax compatibility definitions
    \def\gt{>}
    \def\lt{<}
    \let\Oldtex\TeX
    \let\Oldlatex\LaTeX
    \renewcommand{\TeX}{\textrm{\Oldtex}}
    \renewcommand{\LaTeX}{\textrm{\Oldlatex}}
    % Document parameters
    % Document title
    \title{Proyecto\_seminario}
    
    
    
    
    
% Pygments definitions
\makeatletter
\def\PY@reset{\let\PY@it=\relax \let\PY@bf=\relax%
    \let\PY@ul=\relax \let\PY@tc=\relax%
    \let\PY@bc=\relax \let\PY@ff=\relax}
\def\PY@tok#1{\csname PY@tok@#1\endcsname}
\def\PY@toks#1+{\ifx\relax#1\empty\else%
    \PY@tok{#1}\expandafter\PY@toks\fi}
\def\PY@do#1{\PY@bc{\PY@tc{\PY@ul{%
    \PY@it{\PY@bf{\PY@ff{#1}}}}}}}
\def\PY#1#2{\PY@reset\PY@toks#1+\relax+\PY@do{#2}}

\expandafter\def\csname PY@tok@w\endcsname{\def\PY@tc##1{\textcolor[rgb]{0.73,0.73,0.73}{##1}}}
\expandafter\def\csname PY@tok@c\endcsname{\let\PY@it=\textit\def\PY@tc##1{\textcolor[rgb]{0.25,0.50,0.50}{##1}}}
\expandafter\def\csname PY@tok@cp\endcsname{\def\PY@tc##1{\textcolor[rgb]{0.74,0.48,0.00}{##1}}}
\expandafter\def\csname PY@tok@k\endcsname{\let\PY@bf=\textbf\def\PY@tc##1{\textcolor[rgb]{0.00,0.50,0.00}{##1}}}
\expandafter\def\csname PY@tok@kp\endcsname{\def\PY@tc##1{\textcolor[rgb]{0.00,0.50,0.00}{##1}}}
\expandafter\def\csname PY@tok@kt\endcsname{\def\PY@tc##1{\textcolor[rgb]{0.69,0.00,0.25}{##1}}}
\expandafter\def\csname PY@tok@o\endcsname{\def\PY@tc##1{\textcolor[rgb]{0.40,0.40,0.40}{##1}}}
\expandafter\def\csname PY@tok@ow\endcsname{\let\PY@bf=\textbf\def\PY@tc##1{\textcolor[rgb]{0.67,0.13,1.00}{##1}}}
\expandafter\def\csname PY@tok@nb\endcsname{\def\PY@tc##1{\textcolor[rgb]{0.00,0.50,0.00}{##1}}}
\expandafter\def\csname PY@tok@nf\endcsname{\def\PY@tc##1{\textcolor[rgb]{0.00,0.00,1.00}{##1}}}
\expandafter\def\csname PY@tok@nc\endcsname{\let\PY@bf=\textbf\def\PY@tc##1{\textcolor[rgb]{0.00,0.00,1.00}{##1}}}
\expandafter\def\csname PY@tok@nn\endcsname{\let\PY@bf=\textbf\def\PY@tc##1{\textcolor[rgb]{0.00,0.00,1.00}{##1}}}
\expandafter\def\csname PY@tok@ne\endcsname{\let\PY@bf=\textbf\def\PY@tc##1{\textcolor[rgb]{0.82,0.25,0.23}{##1}}}
\expandafter\def\csname PY@tok@nv\endcsname{\def\PY@tc##1{\textcolor[rgb]{0.10,0.09,0.49}{##1}}}
\expandafter\def\csname PY@tok@no\endcsname{\def\PY@tc##1{\textcolor[rgb]{0.53,0.00,0.00}{##1}}}
\expandafter\def\csname PY@tok@nl\endcsname{\def\PY@tc##1{\textcolor[rgb]{0.63,0.63,0.00}{##1}}}
\expandafter\def\csname PY@tok@ni\endcsname{\let\PY@bf=\textbf\def\PY@tc##1{\textcolor[rgb]{0.60,0.60,0.60}{##1}}}
\expandafter\def\csname PY@tok@na\endcsname{\def\PY@tc##1{\textcolor[rgb]{0.49,0.56,0.16}{##1}}}
\expandafter\def\csname PY@tok@nt\endcsname{\let\PY@bf=\textbf\def\PY@tc##1{\textcolor[rgb]{0.00,0.50,0.00}{##1}}}
\expandafter\def\csname PY@tok@nd\endcsname{\def\PY@tc##1{\textcolor[rgb]{0.67,0.13,1.00}{##1}}}
\expandafter\def\csname PY@tok@s\endcsname{\def\PY@tc##1{\textcolor[rgb]{0.73,0.13,0.13}{##1}}}
\expandafter\def\csname PY@tok@sd\endcsname{\let\PY@it=\textit\def\PY@tc##1{\textcolor[rgb]{0.73,0.13,0.13}{##1}}}
\expandafter\def\csname PY@tok@si\endcsname{\let\PY@bf=\textbf\def\PY@tc##1{\textcolor[rgb]{0.73,0.40,0.53}{##1}}}
\expandafter\def\csname PY@tok@se\endcsname{\let\PY@bf=\textbf\def\PY@tc##1{\textcolor[rgb]{0.73,0.40,0.13}{##1}}}
\expandafter\def\csname PY@tok@sr\endcsname{\def\PY@tc##1{\textcolor[rgb]{0.73,0.40,0.53}{##1}}}
\expandafter\def\csname PY@tok@ss\endcsname{\def\PY@tc##1{\textcolor[rgb]{0.10,0.09,0.49}{##1}}}
\expandafter\def\csname PY@tok@sx\endcsname{\def\PY@tc##1{\textcolor[rgb]{0.00,0.50,0.00}{##1}}}
\expandafter\def\csname PY@tok@m\endcsname{\def\PY@tc##1{\textcolor[rgb]{0.40,0.40,0.40}{##1}}}
\expandafter\def\csname PY@tok@gh\endcsname{\let\PY@bf=\textbf\def\PY@tc##1{\textcolor[rgb]{0.00,0.00,0.50}{##1}}}
\expandafter\def\csname PY@tok@gu\endcsname{\let\PY@bf=\textbf\def\PY@tc##1{\textcolor[rgb]{0.50,0.00,0.50}{##1}}}
\expandafter\def\csname PY@tok@gd\endcsname{\def\PY@tc##1{\textcolor[rgb]{0.63,0.00,0.00}{##1}}}
\expandafter\def\csname PY@tok@gi\endcsname{\def\PY@tc##1{\textcolor[rgb]{0.00,0.63,0.00}{##1}}}
\expandafter\def\csname PY@tok@gr\endcsname{\def\PY@tc##1{\textcolor[rgb]{1.00,0.00,0.00}{##1}}}
\expandafter\def\csname PY@tok@ge\endcsname{\let\PY@it=\textit}
\expandafter\def\csname PY@tok@gs\endcsname{\let\PY@bf=\textbf}
\expandafter\def\csname PY@tok@gp\endcsname{\let\PY@bf=\textbf\def\PY@tc##1{\textcolor[rgb]{0.00,0.00,0.50}{##1}}}
\expandafter\def\csname PY@tok@go\endcsname{\def\PY@tc##1{\textcolor[rgb]{0.53,0.53,0.53}{##1}}}
\expandafter\def\csname PY@tok@gt\endcsname{\def\PY@tc##1{\textcolor[rgb]{0.00,0.27,0.87}{##1}}}
\expandafter\def\csname PY@tok@err\endcsname{\def\PY@bc##1{\setlength{\fboxsep}{0pt}\fcolorbox[rgb]{1.00,0.00,0.00}{1,1,1}{\strut ##1}}}
\expandafter\def\csname PY@tok@kc\endcsname{\let\PY@bf=\textbf\def\PY@tc##1{\textcolor[rgb]{0.00,0.50,0.00}{##1}}}
\expandafter\def\csname PY@tok@kd\endcsname{\let\PY@bf=\textbf\def\PY@tc##1{\textcolor[rgb]{0.00,0.50,0.00}{##1}}}
\expandafter\def\csname PY@tok@kn\endcsname{\let\PY@bf=\textbf\def\PY@tc##1{\textcolor[rgb]{0.00,0.50,0.00}{##1}}}
\expandafter\def\csname PY@tok@kr\endcsname{\let\PY@bf=\textbf\def\PY@tc##1{\textcolor[rgb]{0.00,0.50,0.00}{##1}}}
\expandafter\def\csname PY@tok@bp\endcsname{\def\PY@tc##1{\textcolor[rgb]{0.00,0.50,0.00}{##1}}}
\expandafter\def\csname PY@tok@fm\endcsname{\def\PY@tc##1{\textcolor[rgb]{0.00,0.00,1.00}{##1}}}
\expandafter\def\csname PY@tok@vc\endcsname{\def\PY@tc##1{\textcolor[rgb]{0.10,0.09,0.49}{##1}}}
\expandafter\def\csname PY@tok@vg\endcsname{\def\PY@tc##1{\textcolor[rgb]{0.10,0.09,0.49}{##1}}}
\expandafter\def\csname PY@tok@vi\endcsname{\def\PY@tc##1{\textcolor[rgb]{0.10,0.09,0.49}{##1}}}
\expandafter\def\csname PY@tok@vm\endcsname{\def\PY@tc##1{\textcolor[rgb]{0.10,0.09,0.49}{##1}}}
\expandafter\def\csname PY@tok@sa\endcsname{\def\PY@tc##1{\textcolor[rgb]{0.73,0.13,0.13}{##1}}}
\expandafter\def\csname PY@tok@sb\endcsname{\def\PY@tc##1{\textcolor[rgb]{0.73,0.13,0.13}{##1}}}
\expandafter\def\csname PY@tok@sc\endcsname{\def\PY@tc##1{\textcolor[rgb]{0.73,0.13,0.13}{##1}}}
\expandafter\def\csname PY@tok@dl\endcsname{\def\PY@tc##1{\textcolor[rgb]{0.73,0.13,0.13}{##1}}}
\expandafter\def\csname PY@tok@s2\endcsname{\def\PY@tc##1{\textcolor[rgb]{0.73,0.13,0.13}{##1}}}
\expandafter\def\csname PY@tok@sh\endcsname{\def\PY@tc##1{\textcolor[rgb]{0.73,0.13,0.13}{##1}}}
\expandafter\def\csname PY@tok@s1\endcsname{\def\PY@tc##1{\textcolor[rgb]{0.73,0.13,0.13}{##1}}}
\expandafter\def\csname PY@tok@mb\endcsname{\def\PY@tc##1{\textcolor[rgb]{0.40,0.40,0.40}{##1}}}
\expandafter\def\csname PY@tok@mf\endcsname{\def\PY@tc##1{\textcolor[rgb]{0.40,0.40,0.40}{##1}}}
\expandafter\def\csname PY@tok@mh\endcsname{\def\PY@tc##1{\textcolor[rgb]{0.40,0.40,0.40}{##1}}}
\expandafter\def\csname PY@tok@mi\endcsname{\def\PY@tc##1{\textcolor[rgb]{0.40,0.40,0.40}{##1}}}
\expandafter\def\csname PY@tok@il\endcsname{\def\PY@tc##1{\textcolor[rgb]{0.40,0.40,0.40}{##1}}}
\expandafter\def\csname PY@tok@mo\endcsname{\def\PY@tc##1{\textcolor[rgb]{0.40,0.40,0.40}{##1}}}
\expandafter\def\csname PY@tok@ch\endcsname{\let\PY@it=\textit\def\PY@tc##1{\textcolor[rgb]{0.25,0.50,0.50}{##1}}}
\expandafter\def\csname PY@tok@cm\endcsname{\let\PY@it=\textit\def\PY@tc##1{\textcolor[rgb]{0.25,0.50,0.50}{##1}}}
\expandafter\def\csname PY@tok@cpf\endcsname{\let\PY@it=\textit\def\PY@tc##1{\textcolor[rgb]{0.25,0.50,0.50}{##1}}}
\expandafter\def\csname PY@tok@c1\endcsname{\let\PY@it=\textit\def\PY@tc##1{\textcolor[rgb]{0.25,0.50,0.50}{##1}}}
\expandafter\def\csname PY@tok@cs\endcsname{\let\PY@it=\textit\def\PY@tc##1{\textcolor[rgb]{0.25,0.50,0.50}{##1}}}

\def\PYZbs{\char`\\}
\def\PYZus{\char`\_}
\def\PYZob{\char`\{}
\def\PYZcb{\char`\}}
\def\PYZca{\char`\^}
\def\PYZam{\char`\&}
\def\PYZlt{\char`\<}
\def\PYZgt{\char`\>}
\def\PYZsh{\char`\#}
\def\PYZpc{\char`\%}
\def\PYZdl{\char`\$}
\def\PYZhy{\char`\-}
\def\PYZsq{\char`\'}
\def\PYZdq{\char`\"}
\def\PYZti{\char`\~}
% for compatibility with earlier versions
\def\PYZat{@}
\def\PYZlb{[}
\def\PYZrb{]}
\makeatother


    % For linebreaks inside Verbatim environment from package fancyvrb. 
    \makeatletter
        \newbox\Wrappedcontinuationbox 
        \newbox\Wrappedvisiblespacebox 
        \newcommand*\Wrappedvisiblespace {\textcolor{red}{\textvisiblespace}} 
        \newcommand*\Wrappedcontinuationsymbol {\textcolor{red}{\llap{\tiny$\m@th\hookrightarrow$}}} 
        \newcommand*\Wrappedcontinuationindent {3ex } 
        \newcommand*\Wrappedafterbreak {\kern\Wrappedcontinuationindent\copy\Wrappedcontinuationbox} 
        % Take advantage of the already applied Pygments mark-up to insert 
        % potential linebreaks for TeX processing. 
        %        {, <, #, %, $, ' and ": go to next line. 
        %        _, }, ^, &, >, - and ~: stay at end of broken line. 
        % Use of \textquotesingle for straight quote. 
        \newcommand*\Wrappedbreaksatspecials {% 
            \def\PYGZus{\discretionary{\char`\_}{\Wrappedafterbreak}{\char`\_}}% 
            \def\PYGZob{\discretionary{}{\Wrappedafterbreak\char`\{}{\char`\{}}% 
            \def\PYGZcb{\discretionary{\char`\}}{\Wrappedafterbreak}{\char`\}}}% 
            \def\PYGZca{\discretionary{\char`\^}{\Wrappedafterbreak}{\char`\^}}% 
            \def\PYGZam{\discretionary{\char`\&}{\Wrappedafterbreak}{\char`\&}}% 
            \def\PYGZlt{\discretionary{}{\Wrappedafterbreak\char`\<}{\char`\<}}% 
            \def\PYGZgt{\discretionary{\char`\>}{\Wrappedafterbreak}{\char`\>}}% 
            \def\PYGZsh{\discretionary{}{\Wrappedafterbreak\char`\#}{\char`\#}}% 
            \def\PYGZpc{\discretionary{}{\Wrappedafterbreak\char`\%}{\char`\%}}% 
            \def\PYGZdl{\discretionary{}{\Wrappedafterbreak\char`\$}{\char`\$}}% 
            \def\PYGZhy{\discretionary{\char`\-}{\Wrappedafterbreak}{\char`\-}}% 
            \def\PYGZsq{\discretionary{}{\Wrappedafterbreak\textquotesingle}{\textquotesingle}}% 
            \def\PYGZdq{\discretionary{}{\Wrappedafterbreak\char`\"}{\char`\"}}% 
            \def\PYGZti{\discretionary{\char`\~}{\Wrappedafterbreak}{\char`\~}}% 
        } 
        % Some characters . , ; ? ! / are not pygmentized. 
        % This macro makes them "active" and they will insert potential linebreaks 
        \newcommand*\Wrappedbreaksatpunct {% 
            \lccode`\~`\.\lowercase{\def~}{\discretionary{\hbox{\char`\.}}{\Wrappedafterbreak}{\hbox{\char`\.}}}% 
            \lccode`\~`\,\lowercase{\def~}{\discretionary{\hbox{\char`\,}}{\Wrappedafterbreak}{\hbox{\char`\,}}}% 
            \lccode`\~`\;\lowercase{\def~}{\discretionary{\hbox{\char`\;}}{\Wrappedafterbreak}{\hbox{\char`\;}}}% 
            \lccode`\~`\:\lowercase{\def~}{\discretionary{\hbox{\char`\:}}{\Wrappedafterbreak}{\hbox{\char`\:}}}% 
            \lccode`\~`\?\lowercase{\def~}{\discretionary{\hbox{\char`\?}}{\Wrappedafterbreak}{\hbox{\char`\?}}}% 
            \lccode`\~`\!\lowercase{\def~}{\discretionary{\hbox{\char`\!}}{\Wrappedafterbreak}{\hbox{\char`\!}}}% 
            \lccode`\~`\/\lowercase{\def~}{\discretionary{\hbox{\char`\/}}{\Wrappedafterbreak}{\hbox{\char`\/}}}% 
            \catcode`\.\active
            \catcode`\,\active 
            \catcode`\;\active
            \catcode`\:\active
            \catcode`\?\active
            \catcode`\!\active
            \catcode`\/\active 
            \lccode`\~`\~ 	
        }
    \makeatother

    \let\OriginalVerbatim=\Verbatim
    \makeatletter
    \renewcommand{\Verbatim}[1][1]{%
        %\parskip\z@skip
        \sbox\Wrappedcontinuationbox {\Wrappedcontinuationsymbol}%
        \sbox\Wrappedvisiblespacebox {\FV@SetupFont\Wrappedvisiblespace}%
        \def\FancyVerbFormatLine ##1{\hsize\linewidth
            \vtop{\raggedright\hyphenpenalty\z@\exhyphenpenalty\z@
                \doublehyphendemerits\z@\finalhyphendemerits\z@
                \strut ##1\strut}%
        }%
        % If the linebreak is at a space, the latter will be displayed as visible
        % space at end of first line, and a continuation symbol starts next line.
        % Stretch/shrink are however usually zero for typewriter font.
        \def\FV@Space {%
            \nobreak\hskip\z@ plus\fontdimen3\font minus\fontdimen4\font
            \discretionary{\copy\Wrappedvisiblespacebox}{\Wrappedafterbreak}
            {\kern\fontdimen2\font}%
        }%
        
        % Allow breaks at special characters using \PYG... macros.
        \Wrappedbreaksatspecials
        % Breaks at punctuation characters . , ; ? ! and / need catcode=\active 	
        \OriginalVerbatim[#1,codes*=\Wrappedbreaksatpunct]%
    }
    \makeatother

    % Exact colors from NB
    \definecolor{incolor}{HTML}{303F9F}
    \definecolor{outcolor}{HTML}{D84315}
    \definecolor{cellborder}{HTML}{CFCFCF}
    \definecolor{cellbackground}{HTML}{F7F7F7}
    
    % prompt
    \newcommand{\prompt}[4]{
        \llap{{\color{#2}[#3]: #4}}\vspace{-1.25em}
    }
    

    
    % Prevent overflowing lines due to hard-to-break entities
    \sloppy 
    % Setup hyperref package
    \hypersetup{
      breaklinks=true,  % so long urls are correctly broken across lines
      colorlinks=true,
      urlcolor=urlcolor,
      linkcolor=linkcolor,
      citecolor=citecolor,
      }
    % Slightly bigger margins than the latex defaults
    
    \geometry{verbose,tmargin=1in,bmargin=1in,lmargin=1in,rmargin=1in}
    
    

    \begin{document}
    
    
    \maketitle
    
    

    
    \hypertarget{aplicaciuxf3n-de-un-modelo-oculto-de-muxe1rkov-para-determinar-el-grado-de-honestidad-de-una-persona-al-responder-preguntas.}{%
\section{Aplicación de un modelo oculto de Márkov para determinar el
grado de honestidad de una persona al responder
preguntas.}\label{aplicaciuxf3n-de-un-modelo-oculto-de-muxe1rkov-para-determinar-el-grado-de-honestidad-de-una-persona-al-responder-preguntas.}}

    \hypertarget{dependencias}{%
\subsection{Dependencias:}\label{dependencias}}

Para la realización de este proyecto se necesitaron las siguientes
librerías:

\begin{enumerate}
\def\labelenumi{\arabic{enumi}.}
\tightlist
\item
  numpy: Es un paquete fundamental necesario para la computación
  científica con Python.
\item
  pandas: Es una biblioteca de código abierto que proporciona
  estructuras de datos de alto rendimiento y fáciles de usar, y
  herramientas de análisis de datos.
\item
  NetworkX: Es un paquete de Python para la creación, manipulación y
  estudio de la estructura, dinámica y funciones de redes complejas.
\item
  Matplotlib: Es una biblioteca para hacer diagramas 2D de matrices en
  Python.
\end{enumerate}

    \hypertarget{descripciuxf3n-del-modelo}{%
\subsection{Descripción del
modelo:}\label{descripciuxf3n-del-modelo}}

Este modelo permite descubrir si una persona es honesta por medio de
``señales honestas'' luego de realizarle varias preguntas en una
entrevista.

El modelo está compuesto de tres estados ocultos(deshonesto, insierto y
honesto) con probabilidades iniciales de 1/3 cada uno, cinco estados
observables(cubrirse la boca, mantener una mirada fija, mantener un tono
de voz inestable, tocarse la nariz, rascarse el cuello o cubrir el área
de la garganta), la matriz de transición de estados (MxM), la matriz de
emisión (MxO) y una secuencia de datos de observación.

La idea es determinar cual es el estado oculto mas probable al responder
una pregunta.

    \hypertarget{implementaciuxf3n-del-modelo}{%
\subsection{Implementación del modelo:}\label{implementaciuxf3n-del-modelo}}

    \begin{tcolorbox}[breakable, size=fbox, boxrule=1pt, pad at break*=1mm,colback=cellbackground, colframe=cellborder]
\prompt{In}{incolor}{1}{\hspace{4pt}}
\begin{Verbatim}[commandchars=\\\{\}]
\PY{k+kn}{import} \PY{n+nn}{numpy} \PY{k}{as} \PY{n+nn}{np}
\PY{k+kn}{import} \PY{n+nn}{pandas} \PY{k}{as} \PY{n+nn}{pd}
\PY{k+kn}{import} \PY{n+nn}{networkx} \PY{k}{as} \PY{n+nn}{nx}
\PY{k+kn}{from} \PY{n+nn}{matplotlib} \PY{k}{import} \PY{n}{pyplot} \PY{k}{as} \PY{n}{plt}
\PY{o}{\PYZpc{}}\PY{k}{matplotlib} inline
\PY{k+kn}{from} \PY{n+nn}{pprint} \PY{k}{import} \PY{n}{pprint}
\end{Verbatim}
\end{tcolorbox}

\break
    \begin{tcolorbox}[breakable, size=fbox, boxrule=1pt, pad at break*=1mm,colback=cellbackground, colframe=cellborder]
\prompt{In}{incolor}{2}{\hspace{4pt}}


\begin{Verbatim}[commandchars=\\\{\}]
\PY{k}{def} \PY{n+nf}{viterbi}\PY{p}{(}\PY{n}{pi}\PY{p}{,} \PY{n}{a}\PY{p}{,} \PY{n}{b}\PY{p}{,} \PY{n}{obs}\PY{p}{)}\PY{p}{:}
    
    \PY{n}{nStates} \PY{o}{=} \PY{n}{np}\PY{o}{.}\PY{n}{shape}\PY{p}{(}\PY{n}{b}\PY{p}{)}\PY{p}{[}\PY{l+m+mi}{0}\PY{p}{]}
    \PY{n}{T} \PY{o}{=} \PY{n}{np}\PY{o}{.}\PY{n}{shape}\PY{p}{(}\PY{n}{obs}\PY{p}{)}\PY{p}{[}\PY{l+m+mi}{0}\PY{p}{]}
   
    \PY{c+c1}{\PYZsh{} Camino inicial. Matrices con ceros.}
    \PY{n}{path} \PY{o}{=} \PY{n}{np}\PY{o}{.}\PY{n}{zeros}\PY{p}{(}\PY{n}{T}\PY{p}{)}
    \PY{c+c1}{\PYZsh{} delta \PYZhy{}\PYZhy{}\PYZgt{} mayor probabilidad de cualquier camino que alcance el estado i.}
    \PY{n}{delta} \PY{o}{=} \PY{n}{np}\PY{o}{.}\PY{n}{zeros}\PY{p}{(}\PY{p}{(}\PY{n}{nStates}\PY{p}{,} \PY{n}{T}\PY{p}{)}\PY{p}{)}
    \PY{c+c1}{\PYZsh{} phi \PYZhy{}\PYZhy{}\PYZgt{} argmax por paso de tiempo para cada estado}
    \PY{n}{phi} \PY{o}{=} \PY{n}{np}\PY{o}{.}\PY{n}{zeros}\PY{p}{(}\PY{p}{(}\PY{n}{nStates}\PY{p}{,} \PY{n}{T}\PY{p}{)}\PY{p}{)}
    
    \PY{c+c1}{\PYZsh{} inicio}
    \PY{n}{delta}\PY{p}{[}\PY{p}{:}\PY{p}{,} \PY{l+m+mi}{0}\PY{p}{]} \PY{o}{=} \PY{n}{pi} \PY{o}{*} \PY{n}{b}\PY{p}{[}\PY{p}{:}\PY{p}{,} \PY{n}{obs}\PY{p}{[}\PY{l+m+mi}{0}\PY{p}{]}\PY{p}{]}
    \PY{n}{phi}\PY{p}{[}\PY{p}{:}\PY{p}{,} \PY{l+m+mi}{0}\PY{p}{]} \PY{o}{=} \PY{l+m+mi}{0}

    \PY{n+nb}{print}\PY{p}{(}\PY{l+s+s1}{\PYZsq{}}\PY{l+s+se}{\PYZbs{}n}\PY{l+s+s1}{Inicio. Caminar hacia adelante}\PY{l+s+se}{\PYZbs{}n}\PY{l+s+s1}{\PYZsq{}}\PY{p}{)}    
    \PY{c+c1}{\PYZsh{} Extensión del algoritmo directo}
    \PY{k}{for} \PY{n}{t} \PY{o+ow}{in} \PY{n+nb}{range}\PY{p}{(}\PY{l+m+mi}{1}\PY{p}{,} \PY{n}{T}\PY{p}{)}\PY{p}{:}
        \PY{k}{for} \PY{n}{s} \PY{o+ow}{in} \PY{n+nb}{range}\PY{p}{(}\PY{n}{nStates}\PY{p}{)}\PY{p}{:}
                
            \PY{n}{delta}\PY{p}{[}\PY{n}{s}\PY{p}{,} \PY{n}{t}\PY{p}{]} \PY{o}{=} \PY{n}{np}\PY{o}{.}\PY{n}{max}\PY{p}{(}\PY{n}{delta}\PY{p}{[}\PY{p}{:}\PY{p}{,} \PY{n}{t}\PY{o}{\PYZhy{}}\PY{l+m+mi}{1}\PY{p}{]} \PY{o}{*} \PY{n}{a}\PY{p}{[}\PY{p}{:}\PY{p}{,} \PY{n}{s}\PY{p}{]}\PY{p}{)} \PY{o}{*} \PY{n}{b}\PY{p}{[}\PY{n}{s}\PY{p}{,} \PY{n}{obs}\PY{p}{[}\PY{n}{t}\PY{p}{]}\PY{p}{]} 
            \PY{n}{phi}\PY{p}{[}\PY{n}{s}\PY{p}{,} \PY{n}{t}\PY{p}{]} \PY{o}{=} \PY{n}{np}\PY{o}{.}\PY{n}{argmax}\PY{p}{(}\PY{n}{delta}\PY{p}{[}\PY{p}{:}\PY{p}{,} \PY{n}{t}\PY{o}{\PYZhy{}}\PY{l+m+mi}{1}\PY{p}{]} \PY{o}{*} \PY{n}{a}\PY{p}{[}\PY{p}{:}\PY{p}{,} \PY{n}{s}\PY{p}{]}\PY{p}{)}
            \PY{n+nb}{print}\PY{p}{(}\PY{l+s+s1}{\PYZsq{}}\PY{l+s+s1}{s=}\PY{l+s+si}{\PYZob{}s\PYZcb{}}\PY{l+s+s1}{ and t=}\PY{l+s+si}{\PYZob{}t\PYZcb{}}\PY{l+s+s1}{: phi[}\PY{l+s+si}{\PYZob{}s\PYZcb{}}\PY{l+s+s1}{, }\PY{l+s+si}{\PYZob{}t\PYZcb{}}\PY{l+s+s1}{] = }\PY{l+s+si}{\PYZob{}phi\PYZcb{}}\PY{l+s+s1}{\PYZsq{}}\PY{o}{.}\PY{n}{format}\PY{p}{(}\PY{n}{s}\PY{o}{=}\PY{n}{s}\PY{p}{,} \PY{n}{t}\PY{o}{=}\PY{n}{t}\PY{p}{,} \PY{n}{phi}\PY{o}{=}\PY{n}{phi}\PY{p}{[}\PY{n}{s}\PY{p}{,} \PY{n}{t}\PY{p}{]}\PY{p}{)}\PY{p}{)}
    
    \PY{c+c1}{\PYZsh{} Encontrar el camino óptimo}
    \PY{n+nb}{print}\PY{p}{(}\PY{l+s+s1}{\PYZsq{}}\PY{l+s+s1}{\PYZhy{}}\PY{l+s+s1}{\PYZsq{}}\PY{o}{*}\PY{l+m+mi}{50}\PY{p}{)}
    \PY{n+nb}{print}\PY{p}{(}\PY{l+s+s1}{\PYZsq{}}\PY{l+s+s1}{Iniciar retroceso}\PY{l+s+se}{\PYZbs{}n}\PY{l+s+s1}{\PYZsq{}}\PY{p}{)}
    \PY{n}{path}\PY{p}{[}\PY{n}{T}\PY{o}{\PYZhy{}}\PY{l+m+mi}{1}\PY{p}{]} \PY{o}{=} \PY{n}{np}\PY{o}{.}\PY{n}{argmax}\PY{p}{(}\PY{n}{delta}\PY{p}{[}\PY{p}{:}\PY{p}{,} \PY{n}{T}\PY{o}{\PYZhy{}}\PY{l+m+mi}{1}\PY{p}{]}\PY{p}{)}
    
    
    \PY{k}{for} \PY{n}{t} \PY{o+ow}{in} \PY{n+nb}{range}\PY{p}{(}\PY{n}{T}\PY{o}{\PYZhy{}}\PY{l+m+mi}{2}\PY{p}{,} \PY{o}{\PYZhy{}}\PY{l+m+mi}{1}\PY{p}{,} \PY{o}{\PYZhy{}}\PY{l+m+mi}{1}\PY{p}{)}\PY{p}{:}
        
        \PY{n}{x} \PY{o}{=} \PY{n+nb}{int}\PY{p}{(}\PY{n}{path}\PY{p}{[}\PY{n}{t}\PY{o}{+}\PY{l+m+mi}{1}\PY{p}{]}\PY{p}{)}
    
        \PY{n}{path}\PY{p}{[}\PY{n}{t}\PY{p}{]} \PY{o}{=} \PY{n}{phi}\PY{p}{[}\PY{n}{x}\PY{p}{,} \PY{p}{[}\PY{n}{t}\PY{o}{+}\PY{l+m+mi}{1}\PY{p}{]}\PY{p}{]}
        \PY{n+nb}{print}\PY{p}{(}\PY{l+s+s1}{\PYZsq{}}\PY{l+s+s1}{path[}\PY{l+s+si}{\PYZob{}\PYZcb{}}\PY{l+s+s1}{] = }\PY{l+s+si}{\PYZob{}\PYZcb{}}\PY{l+s+s1}{\PYZsq{}}\PY{o}{.}\PY{n}{format}\PY{p}{(}\PY{n}{t}\PY{p}{,} \PY{n}{path}\PY{p}{[}\PY{n}{t}\PY{p}{]}\PY{p}{)}\PY{p}{)}
        
    \PY{k}{return} \PY{n}{path}\PY{p}{,} \PY{n}{delta}\PY{p}{,} \PY{n}{phi}
\end{Verbatim}
\end{tcolorbox}

    \hypertarget{descripciuxf3n-del-entrenamiento}{%
\subsection{Descripción del
entrenamiento:}\label{descripciuxf3n-del-entrenamiento}}

    Se utilizó el algoritmo de Baum-Welch para definir la matriz de
transición de estados ocultos y la matriz de emision u observación. Este
algoritmo de entrenamiento necesita como parámetros los estados
observables, los estados ocultos y la probabilidad inicial de los
estados ocultos. Luego estas matrices se pasan como parámetros al
algoritmo de Viterbi junto con la probabilidad inicial de los estados
ocultos para conseguir la secuencia mas probable de estados ocultos
(deshonesto, neutral y honesto) que es producida por una secuencia de
estados observables(cubrirse la boca, mantener una mirada fija, mantener
un tono de voz inestable, tocarse la nariz y cubrir el área de la
garganta) con el fin de determinar si la persona entrevistada esta
siendo honesta con sus respuestas.

    \begin{tcolorbox}[breakable, size=fbox, boxrule=1pt, pad at break*=1mm,colback=cellbackground, colframe=cellborder]
\prompt{In}{incolor}{3}{\hspace{4pt}}
\begin{Verbatim}[commandchars=\\\{\}]
\PY{k}{def} \PY{n+nf}{forward}\PY{p}{(}\PY{n}{V}\PY{p}{,} \PY{n}{a}\PY{p}{,} \PY{n}{b}\PY{p}{,} \PY{n}{initial\PYZus{}distribution}\PY{p}{)}\PY{p}{:}
    \PY{n}{alpha} \PY{o}{=} \PY{n}{np}\PY{o}{.}\PY{n}{zeros}\PY{p}{(}\PY{p}{(}\PY{n}{V}\PY{o}{.}\PY{n}{shape}\PY{p}{[}\PY{l+m+mi}{0}\PY{p}{]}\PY{p}{,} \PY{n}{a}\PY{o}{.}\PY{n}{shape}\PY{p}{[}\PY{l+m+mi}{0}\PY{p}{]}\PY{p}{)}\PY{p}{)}
    \PY{n}{alpha}\PY{p}{[}\PY{l+m+mi}{0}\PY{p}{,} \PY{p}{:}\PY{p}{]} \PY{o}{=} \PY{n}{initial\PYZus{}distribution} \PY{o}{*} \PY{n}{b}\PY{p}{[}\PY{p}{:}\PY{p}{,} \PY{n}{V}\PY{p}{[}\PY{l+m+mi}{0}\PY{p}{]}\PY{p}{]}

    \PY{k}{for} \PY{n}{t} \PY{o+ow}{in} \PY{n+nb}{range}\PY{p}{(}\PY{l+m+mi}{1}\PY{p}{,} \PY{n}{V}\PY{o}{.}\PY{n}{shape}\PY{p}{[}\PY{l+m+mi}{0}\PY{p}{]}\PY{p}{)}\PY{p}{:}
        \PY{k}{for} \PY{n}{j} \PY{o+ow}{in} \PY{n+nb}{range}\PY{p}{(}\PY{n}{a}\PY{o}{.}\PY{n}{shape}\PY{p}{[}\PY{l+m+mi}{0}\PY{p}{]}\PY{p}{)}\PY{p}{:}
            \PY{c+c1}{\PYZsh{} Matrix Computation Steps}
            \PY{c+c1}{\PYZsh{}                  ((1x2) . (1x2))      *     (1)}
            \PY{c+c1}{\PYZsh{}                        (1)            *     (1)}
            \PY{n}{alpha}\PY{p}{[}\PY{n}{t}\PY{p}{,} \PY{n}{j}\PY{p}{]} \PY{o}{=} \PY{n}{alpha}\PY{p}{[}\PY{n}{t} \PY{o}{\PYZhy{}} \PY{l+m+mi}{1}\PY{p}{]}\PY{o}{.}\PY{n}{dot}\PY{p}{(}\PY{n}{a}\PY{p}{[}\PY{p}{:}\PY{p}{,} \PY{n}{j}\PY{p}{]}\PY{p}{)} \PY{o}{*} \PY{n}{b}\PY{p}{[}\PY{n}{j}\PY{p}{,} \PY{n}{V}\PY{p}{[}\PY{n}{t}\PY{p}{]}\PY{p}{]}
            
    \PY{k}{return} \PY{n}{alpha}
 

\PY{k}{def} \PY{n+nf}{backward}\PY{p}{(}\PY{n}{V}\PY{p}{,} \PY{n}{a}\PY{p}{,} \PY{n}{b}\PY{p}{)}\PY{p}{:}
    \PY{n}{beta} \PY{o}{=} \PY{n}{np}\PY{o}{.}\PY{n}{zeros}\PY{p}{(}\PY{p}{(}\PY{n}{V}\PY{o}{.}\PY{n}{shape}\PY{p}{[}\PY{l+m+mi}{0}\PY{p}{]}\PY{p}{,} \PY{n}{a}\PY{o}{.}\PY{n}{shape}\PY{p}{[}\PY{l+m+mi}{0}\PY{p}{]}\PY{p}{)}\PY{p}{)}
    
    \PY{c+c1}{\PYZsh{} setting beta(T) = 1}
    \PY{n}{beta}\PY{p}{[}\PY{n}{V}\PY{o}{.}\PY{n}{shape}\PY{p}{[}\PY{l+m+mi}{0}\PY{p}{]} \PY{o}{\PYZhy{}} \PY{l+m+mi}{1}\PY{p}{]} \PY{o}{=} \PY{n}{np}\PY{o}{.}\PY{n}{ones}\PY{p}{(}\PY{p}{(}\PY{n}{a}\PY{o}{.}\PY{n}{shape}\PY{p}{[}\PY{l+m+mi}{0}\PY{p}{]}\PY{p}{)}\PY{p}{)}

    \PY{c+c1}{\PYZsh{} Loop in backward way from T\PYZhy{}1 to}
    \PY{c+c1}{\PYZsh{} Due to python indexing the actual loop will be T\PYZhy{}2 to 0}
    \PY{k}{for} \PY{n}{t} \PY{o+ow}{in} \PY{n+nb}{range}\PY{p}{(}\PY{n}{V}\PY{o}{.}\PY{n}{shape}\PY{p}{[}\PY{l+m+mi}{0}\PY{p}{]} \PY{o}{\PYZhy{}} \PY{l+m+mi}{2}\PY{p}{,} \PY{o}{\PYZhy{}}\PY{l+m+mi}{1}\PY{p}{,} \PY{o}{\PYZhy{}}\PY{l+m+mi}{1}\PY{p}{)}\PY{p}{:}
        \PY{k}{for} \PY{n}{j} \PY{o+ow}{in} \PY{n+nb}{range}\PY{p}{(}\PY{n}{a}\PY{o}{.}\PY{n}{shape}\PY{p}{[}\PY{l+m+mi}{0}\PY{p}{]}\PY{p}{)}\PY{p}{:}
            \PY{n}{beta}\PY{p}{[}\PY{n}{t}\PY{p}{,} \PY{n}{j}\PY{p}{]} \PY{o}{=} \PY{p}{(}\PY{n}{beta}\PY{p}{[}\PY{n}{t} \PY{o}{+} \PY{l+m+mi}{1}\PY{p}{]} \PY{o}{*} \PY{n}{b}\PY{p}{[}\PY{p}{:}\PY{p}{,} \PY{n}{V}\PY{p}{[}\PY{n}{t} \PY{o}{+} \PY{l+m+mi}{1}\PY{p}{]}\PY{p}{]}\PY{p}{)}\PY{o}{.}\PY{n}{dot}\PY{p}{(}\PY{n}{a}\PY{p}{[}\PY{n}{j}\PY{p}{,} \PY{p}{:}\PY{p}{]}\PY{p}{)}
    \PY{k}{return} \PY{n}{beta}
 

\PY{k}{def} \PY{n+nf}{baum\PYZus{}welch}\PY{p}{(}\PY{n}{V}\PY{p}{,} \PY{n}{a}\PY{p}{,} \PY{n}{b}\PY{p}{,} \PY{n}{initial\PYZus{}distribution}\PY{p}{,} \PY{n}{n\PYZus{}iter}\PY{o}{=}\PY{l+m+mi}{100}\PY{p}{)}\PY{p}{:}
    \PY{n}{M} \PY{o}{=} \PY{n}{a}\PY{o}{.}\PY{n}{shape}\PY{p}{[}\PY{l+m+mi}{0}\PY{p}{]}
    \PY{n}{T} \PY{o}{=} \PY{n+nb}{len}\PY{p}{(}\PY{n}{V}\PY{p}{)}

    \PY{k}{for} \PY{n}{n} \PY{o+ow}{in} \PY{n+nb}{range}\PY{p}{(}\PY{n}{n\PYZus{}iter}\PY{p}{)}\PY{p}{:}
        \PY{n}{alpha} \PY{o}{=} \PY{n}{forward}\PY{p}{(}\PY{n}{V}\PY{p}{,} \PY{n}{a}\PY{p}{,} \PY{n}{b}\PY{p}{,} \PY{n}{initial\PYZus{}distribution}\PY{p}{)}
        \PY{n}{beta} \PY{o}{=} \PY{n}{backward}\PY{p}{(}\PY{n}{V}\PY{p}{,} \PY{n}{a}\PY{p}{,} \PY{n}{b}\PY{p}{)}

        \PY{n}{xi} \PY{o}{=} \PY{n}{np}\PY{o}{.}\PY{n}{zeros}\PY{p}{(}\PY{p}{(}\PY{n}{M}\PY{p}{,} \PY{n}{M}\PY{p}{,} \PY{n}{T} \PY{o}{\PYZhy{}} \PY{l+m+mi}{1}\PY{p}{)}\PY{p}{)}
        \PY{k}{for} \PY{n}{t} \PY{o+ow}{in} \PY{n+nb}{range}\PY{p}{(}\PY{n}{T} \PY{o}{\PYZhy{}} \PY{l+m+mi}{1}\PY{p}{)}\PY{p}{:}
            \PY{n}{denominator} \PY{o}{=} \PY{n}{np}\PY{o}{.}\PY{n}{dot}\PY{p}{(}\PY{n}{np}\PY{o}{.}\PY{n}{dot}\PY{p}{(}\PY{n}{alpha}\PY{p}{[}\PY{n}{t}\PY{p}{,} \PY{p}{:}\PY{p}{]}\PY{o}{.}\PY{n}{T}\PY{p}{,} \PY{n}{a}\PY{p}{)} \PY{o}{*} \PY{n}{b}\PY{p}{[}\PY{p}{:}\PY{p}{,} \PY{n}{V}\PY{p}{[}\PY{n}{t} \PY{o}{+} \PY{l+m+mi}{1}\PY{p}{]}\PY{p}{]}\PY{o}{.}\PY{n}{T}\PY{p}{,} \PY{n}{beta}\PY{p}{[}\PY{n}{t} \PY{o}{+} \PY{l+m+mi}{1}\PY{p}{,} \PY{p}{:}\PY{p}{]}\PY{p}{)}
            \PY{k}{for} \PY{n}{i} \PY{o+ow}{in} \PY{n+nb}{range}\PY{p}{(}\PY{n}{M}\PY{p}{)}\PY{p}{:}
                \PY{n}{numerator} \PY{o}{=} \PY{n}{alpha}\PY{p}{[}\PY{n}{t}\PY{p}{,} \PY{n}{i}\PY{p}{]} \PY{o}{*} \PY{n}{a}\PY{p}{[}\PY{n}{i}\PY{p}{,} \PY{p}{:}\PY{p}{]} \PY{o}{*} \PY{n}{b}\PY{p}{[}\PY{p}{:}\PY{p}{,} \PY{n}{V}\PY{p}{[}\PY{n}{t} \PY{o}{+} \PY{l+m+mi}{1}\PY{p}{]}\PY{p}{]}\PY{o}{.}\PY{n}{T} \PY{o}{*} \PY{n}{beta}\PY{p}{[}\PY{n}{t} \PY{o}{+} \PY{l+m+mi}{1}\PY{p}{,} \PY{p}{:}\PY{p}{]}\PY{o}{.}\PY{n}{T}
                \PY{n}{xi}\PY{p}{[}\PY{n}{i}\PY{p}{,} \PY{p}{:}\PY{p}{,} \PY{n}{t}\PY{p}{]} \PY{o}{=} \PY{n}{numerator} \PY{o}{/} \PY{n}{denominator}
                
        \PY{n}{gamma} \PY{o}{=} \PY{n}{np}\PY{o}{.}\PY{n}{sum}\PY{p}{(}\PY{n}{xi}\PY{p}{,} \PY{n}{axis}\PY{o}{=}\PY{l+m+mi}{1}\PY{p}{)}
        \PY{n}{a} \PY{o}{=} \PY{n}{np}\PY{o}{.}\PY{n}{sum}\PY{p}{(}\PY{n}{xi}\PY{p}{,} \PY{l+m+mi}{2}\PY{p}{)} \PY{o}{/} \PY{n}{np}\PY{o}{.}\PY{n}{sum}\PY{p}{(}\PY{n}{gamma}\PY{p}{,} \PY{n}{axis}\PY{o}{=}\PY{l+m+mi}{1}\PY{p}{)}\PY{o}{.}\PY{n}{reshape}\PY{p}{(}\PY{p}{(}\PY{o}{\PYZhy{}}\PY{l+m+mi}{1}\PY{p}{,} \PY{l+m+mi}{1}\PY{p}{)}\PY{p}{)}

        \PY{c+c1}{\PYZsh{} Add additional T\PYZsq{}th element in gamma}
        \PY{n}{gamma} \PY{o}{=} \PY{n}{np}\PY{o}{.}\PY{n}{hstack}\PY{p}{(}\PY{p}{(}\PY{n}{gamma}\PY{p}{,} \PY{n}{np}\PY{o}{.}\PY{n}{sum}\PY{p}{(}\PY{n}{xi}\PY{p}{[}\PY{p}{:}\PY{p}{,} \PY{p}{:}\PY{p}{,} \PY{n}{T} \PY{o}{\PYZhy{}} \PY{l+m+mi}{2}\PY{p}{]}\PY{p}{,} \PY{n}{axis}\PY{o}{=}\PY{l+m+mi}{0}\PY{p}{)}\PY{o}{.}\PY{n}{reshape}\PY{p}{(}\PY{p}{(}\PY{o}{\PYZhy{}}\PY{l+m+mi}{1}\PY{p}{,} \PY{l+m+mi}{1}\PY{p}{)}\PY{p}{)}\PY{p}{)}\PY{p}{)}

        \PY{n}{K} \PY{o}{=} \PY{n}{b}\PY{o}{.}\PY{n}{shape}\PY{p}{[}\PY{l+m+mi}{1}\PY{p}{]}
        \PY{n}{denominator} \PY{o}{=} \PY{n}{np}\PY{o}{.}\PY{n}{sum}\PY{p}{(}\PY{n}{gamma}\PY{p}{,} \PY{n}{axis}\PY{o}{=}\PY{l+m+mi}{1}\PY{p}{)}
        \PY{k}{for} \PY{n}{l} \PY{o+ow}{in} \PY{n+nb}{range}\PY{p}{(}\PY{n}{K}\PY{p}{)}\PY{p}{:}
            \PY{n}{b}\PY{p}{[}\PY{p}{:}\PY{p}{,} \PY{n}{l}\PY{p}{]} \PY{o}{=} \PY{n}{np}\PY{o}{.}\PY{n}{sum}\PY{p}{(}\PY{n}{gamma}\PY{p}{[}\PY{p}{:}\PY{p}{,} \PY{n}{V} \PY{o}{==} \PY{n}{l}\PY{p}{]}\PY{p}{,} \PY{n}{axis}\PY{o}{=}\PY{l+m+mi}{1}\PY{p}{)}

        \PY{n}{b} \PY{o}{=} \PY{n}{np}\PY{o}{.}\PY{n}{divide}\PY{p}{(}\PY{n}{b}\PY{p}{,} \PY{n}{denominator}\PY{o}{.}\PY{n}{reshape}\PY{p}{(}\PY{p}{(}\PY{o}{\PYZhy{}}\PY{l+m+mi}{1}\PY{p}{,} \PY{l+m+mi}{1}\PY{p}{)}\PY{p}{)}\PY{p}{)}
        

    \PY{k}{return} \PY{n}{a}\PY{p}{,} \PY{n}{b}
\end{Verbatim}
\end{tcolorbox}

    Se pasan los parametros al modelo de entrenamiento:

    \begin{tcolorbox}[breakable, size=fbox, boxrule=1pt, pad at break*=1mm,colback=cellbackground, colframe=cellborder]
\prompt{In}{incolor}{4}{\hspace{4pt}}
\begin{Verbatim}[commandchars=\\\{\}]
\PY{n}{data} \PY{o}{=} \PY{n}{pd}\PY{o}{.}\PY{n}{read\PYZus{}csv}\PY{p}{(}\PY{l+s+s1}{\PYZsq{}}\PY{l+s+s1}{/home/katherine/compu\PYZus{}kathe/9no\PYZus{}semestre/seminario/proyecto\PYZus{}seminario/prueba1.csv}\PY{l+s+s1}{\PYZsq{}}\PY{p}{)}

\PY{n}{V} \PY{o}{=} \PY{n}{data}\PY{p}{[}\PY{l+s+s1}{\PYZsq{}}\PY{l+s+s1}{visible}\PY{l+s+s1}{\PYZsq{}}\PY{p}{]}\PY{o}{.}\PY{n}{values}
\PY{n}{obs\PYZus{}map} \PY{o}{=} \PY{p}{\PYZob{}}\PY{l+s+s1}{\PYZsq{}}\PY{l+s+s1}{CB}\PY{l+s+s1}{\PYZsq{}}\PY{p}{:}\PY{l+m+mi}{0}\PY{p}{,} \PY{l+s+s1}{\PYZsq{}}\PY{l+s+s1}{TN}\PY{l+s+s1}{\PYZsq{}}\PY{p}{:}\PY{l+m+mi}{1}\PY{p}{,} \PY{l+s+s1}{\PYZsq{}}\PY{l+s+s1}{CG}\PY{l+s+s1}{\PYZsq{}}\PY{p}{:}\PY{l+m+mi}{2}\PY{p}{,} \PY{l+s+s1}{\PYZsq{}}\PY{l+s+s1}{MF}\PY{l+s+s1}{\PYZsq{}}\PY{p}{:}\PY{l+m+mi}{3}\PY{p}{,} \PY{l+s+s1}{\PYZsq{}}\PY{l+s+s1}{VI}\PY{l+s+s1}{\PYZsq{}}\PY{p}{:}\PY{l+m+mi}{4}\PY{p}{,} \PY{l+s+s1}{\PYZsq{}}\PY{l+s+s1}{NO}\PY{l+s+s1}{\PYZsq{}}\PY{p}{:}\PY{l+m+mi}{5}\PY{p}{\PYZcb{}}
\PY{c+c1}{\PYZsh{}CB = cubrir boca, MF = Mirada fija, VI= tono voz inestable, CG = cubrirse garganta TN = tocarse la nariz, }
\PY{c+c1}{\PYZsh{}NO = niguna de las anteriores.  }

\PY{n}{array} \PY{o}{=} \PY{p}{[}\PY{p}{]}

\PY{k}{for} \PY{n}{i} \PY{o+ow}{in} \PY{n}{V}\PY{p}{:}
    \PY{c+c1}{\PYZsh{}print(\PYZsq{}\PYZbs{}n i: \PYZsq{},i)}
    \PY{k}{for} \PY{n}{k}\PY{p}{,} \PY{n}{v} \PY{o+ow}{in} \PY{n}{obs\PYZus{}map}\PY{o}{.}\PY{n}{items}\PY{p}{(}\PY{p}{)}\PY{p}{:}
       \PY{c+c1}{\PYZsh{} print(\PYZsq{}k : \PYZsq{},k,\PYZsq{} v: \PYZsq{}, v)}
        \PY{k}{if} \PY{n}{i} \PY{o}{==} \PY{n}{k}\PY{p}{:}
            \PY{n}{array}\PY{o}{.}\PY{n}{append}\PY{p}{(}\PY{n}{v}\PY{p}{)}

\PY{n}{obs} \PY{o}{=} \PY{n}{np}\PY{o}{.}\PY{n}{array}\PY{p}{(}\PY{n}{array}\PY{p}{)}

\PY{c+c1}{\PYZsh{} Transition Probabilities}
\PY{n}{a} \PY{o}{=} \PY{n}{np}\PY{o}{.}\PY{n}{ones}\PY{p}{(}\PY{p}{(}\PY{l+m+mi}{3}\PY{p}{,} \PY{l+m+mi}{3}\PY{p}{)}\PY{p}{)}
\PY{n}{a} \PY{o}{=} \PY{n}{a} \PY{o}{/} \PY{n}{np}\PY{o}{.}\PY{n}{sum}\PY{p}{(}\PY{n}{a}\PY{p}{,} \PY{n}{axis}\PY{o}{=}\PY{l+m+mi}{1}\PY{p}{)}

\PY{c+c1}{\PYZsh{}Estas son las probabilidades de estado inicial}
\PY{n}{pi} \PY{o}{=} \PY{n}{a}\PY{p}{[}\PY{l+m+mi}{0}\PY{p}{]}

\PY{c+c1}{\PYZsh{} Emission Probabilities}
\PY{n}{b} \PY{o}{=} \PY{n}{np}\PY{o}{.}\PY{n}{array}\PY{p}{(}\PY{p}{(}\PY{p}{(}\PY{l+m+mi}{1}\PY{p}{,} \PY{l+m+mi}{3}\PY{p}{,} \PY{l+m+mi}{5}\PY{p}{,} \PY{l+m+mi}{1}\PY{p}{,} \PY{l+m+mi}{3}\PY{p}{,} \PY{l+m+mi}{5}\PY{p}{)}\PY{p}{,} \PY{p}{(}\PY{l+m+mi}{2}\PY{p}{,} \PY{l+m+mi}{4}\PY{p}{,} \PY{l+m+mi}{6}\PY{p}{,} \PY{l+m+mi}{2}\PY{p}{,} \PY{l+m+mi}{4}\PY{p}{,} \PY{l+m+mi}{6}\PY{p}{)}\PY{p}{,}\PY{p}{(}\PY{l+m+mi}{1}\PY{p}{,} \PY{l+m+mi}{3}\PY{p}{,} \PY{l+m+mi}{5}\PY{p}{,} \PY{l+m+mi}{1}\PY{p}{,} \PY{l+m+mi}{3}\PY{p}{,} \PY{l+m+mi}{5}\PY{p}{)}\PY{p}{)}\PY{p}{)}
\PY{c+c1}{\PYZsh{}print(\PYZsq{}B: \PYZsq{},b)}
\PY{n}{b} \PY{o}{=} \PY{n}{b} \PY{o}{/} \PY{n}{np}\PY{o}{.}\PY{n}{sum}\PY{p}{(}\PY{n}{b}\PY{p}{,} \PY{n}{axis}\PY{o}{=}\PY{l+m+mi}{1}\PY{p}{)}\PY{o}{.}\PY{n}{reshape}\PY{p}{(}\PY{p}{(}\PY{o}{\PYZhy{}}\PY{l+m+mi}{1}\PY{p}{,} \PY{l+m+mi}{1}\PY{p}{)}\PY{p}{)}

\PY{n}{a}\PY{p}{,} \PY{n}{b}\PY{o}{=} \PY{n}{baum\PYZus{}welch}\PY{p}{(}\PY{n}{obs}\PY{p}{,} \PY{n}{a}\PY{p}{,} \PY{n}{b}\PY{p}{,} \PY{n}{pi}\PY{p}{,} \PY{n}{n\PYZus{}iter}\PY{o}{=}\PY{l+m+mi}{100}\PY{p}{)}

\PY{n+nb}{print}\PY{p}{(}\PY{l+s+s1}{\PYZsq{}}\PY{l+s+s1}{MATRIZ A: }\PY{l+s+s1}{\PYZsq{}}\PY{p}{,} \PY{n}{a}\PY{p}{,} \PY{l+s+s1}{\PYZsq{}}\PY{l+s+se}{\PYZbs{}n}\PY{l+s+s1}{\PYZsq{}}\PY{p}{)}
\PY{n+nb}{print}\PY{p}{(}\PY{l+s+s1}{\PYZsq{}}\PY{l+s+s1}{MATRIZ B: }\PY{l+s+s1}{\PYZsq{}}\PY{p}{,} \PY{n}{b}\PY{p}{,} \PY{l+s+s1}{\PYZsq{}}\PY{l+s+se}{\PYZbs{}n}\PY{l+s+s1}{\PYZsq{}}\PY{p}{)}
\end{Verbatim}
\end{tcolorbox}

    \begin{Verbatim}[commandchars=\\\{\}]
MATRIZ A:  [[2.09834444e-01 5.80331112e-01 2.09834444e-01]
 [2.65586933e-05 9.99946883e-01 2.65586933e-05]
 [2.09834444e-01 5.80331112e-01 2.09834444e-01]]

MATRIZ B:  [[9.56372350e-132 1.00000000e+000 1.28337566e-010 4.23819947e-131
  1.41836799e-132 8.10545154e-094]
 [1.19903231e-001 1.60677384e-001 2.39806462e-001 1.19903231e-001
  1.19903231e-001 2.39806462e-001]
 [9.56372350e-132 1.00000000e+000 1.28337566e-010 4.23819947e-131
  1.41836799e-132 8.10545154e-094]]

\end{Verbatim}

    Se muestran las matrices de transición y observación

    \begin{tcolorbox}[breakable, size=fbox, boxrule=1pt, pad at break*=1mm,colback=cellbackground, colframe=cellborder]
\prompt{In}{incolor}{5}{\hspace{4pt}}
\begin{Verbatim}[commandchars=\\\{\}]
\PY{c+c1}{\PYZsh{}Se asigna a un arreglo los estados ocultos}
\PY{n+nb}{print}\PY{p}{(}\PY{l+s+s1}{\PYZsq{}}\PY{l+s+s1}{Probabilidades Iniciales de los Estados Ocultos: }\PY{l+s+se}{\PYZbs{}n}\PY{l+s+s1}{\PYZsq{}}\PY{p}{)}
\PY{n}{hidden\PYZus{}states} \PY{o}{=} \PY{p}{[}\PY{l+s+s1}{\PYZsq{}}\PY{l+s+s1}{deshonesto}\PY{l+s+s1}{\PYZsq{}}\PY{p}{,} \PY{l+s+s1}{\PYZsq{}}\PY{l+s+s1}{incierto}\PY{l+s+s1}{\PYZsq{}}\PY{p}{,} \PY{l+s+s1}{\PYZsq{}}\PY{l+s+s1}{honesto}\PY{l+s+s1}{\PYZsq{}}\PY{p}{]}

\PY{n}{state\PYZus{}space} \PY{o}{=} \PY{n}{pd}\PY{o}{.}\PY{n}{Series}\PY{p}{(}\PY{n}{pi}\PY{p}{,} \PY{n}{index}\PY{o}{=}\PY{n}{hidden\PYZus{}states}\PY{p}{,} \PY{n}{name}\PY{o}{=}\PY{l+s+s1}{\PYZsq{}}\PY{l+s+s1}{Estados}\PY{l+s+s1}{\PYZsq{}}\PY{p}{)}
\PY{n+nb}{print}\PY{p}{(}\PY{n}{state\PYZus{}space}\PY{p}{)}
\PY{n+nb}{print}\PY{p}{(}\PY{l+s+s1}{\PYZsq{}}\PY{l+s+se}{\PYZbs{}n}\PY{l+s+s1}{\PYZsq{}}\PY{p}{,} \PY{n}{state\PYZus{}space}\PY{o}{.}\PY{n}{sum}\PY{p}{(}\PY{p}{)}\PY{p}{)}
\PY{n+nb}{print}\PY{p}{(}\PY{l+s+s1}{\PYZsq{}}\PY{l+s+se}{\PYZbs{}n}\PY{l+s+s1}{\PYZsq{}}\PY{p}{)}

\PY{c+c1}{\PYZsh{}Matriz de probabilidad de transición de estados cambiantes dado un estado}
\PY{c+c1}{\PYZsh{}Se crea una matriz (MxM) donde M es el número de estados.}

\PY{n+nb}{print}\PY{p}{(}\PY{l+s+s1}{\PYZsq{}}\PY{l+s+s1}{Matriz de Transición De modelo: }\PY{l+s+se}{\PYZbs{}n}\PY{l+s+s1}{\PYZsq{}}\PY{p}{)}
\PY{n}{a\PYZus{}df} \PY{o}{=} \PY{n}{pd}\PY{o}{.}\PY{n}{DataFrame}\PY{p}{(}\PY{n}{columns}\PY{o}{=}\PY{n}{hidden\PYZus{}states}\PY{p}{,} \PY{n}{index}\PY{o}{=}\PY{n}{hidden\PYZus{}states}\PY{p}{)}
\PY{n}{a\PYZus{}df}\PY{o}{.}\PY{n}{loc}\PY{p}{[}\PY{n}{hidden\PYZus{}states}\PY{p}{[}\PY{l+m+mi}{0}\PY{p}{]}\PY{p}{]} \PY{o}{=} \PY{n}{a}\PY{p}{[}\PY{l+m+mi}{0}\PY{p}{]}
\PY{n}{a\PYZus{}df}\PY{o}{.}\PY{n}{loc}\PY{p}{[}\PY{n}{hidden\PYZus{}states}\PY{p}{[}\PY{l+m+mi}{1}\PY{p}{]}\PY{p}{]} \PY{o}{=} \PY{n}{a}\PY{p}{[}\PY{l+m+mi}{1}\PY{p}{]}
\PY{n}{a\PYZus{}df}\PY{o}{.}\PY{n}{loc}\PY{p}{[}\PY{n}{hidden\PYZus{}states}\PY{p}{[}\PY{l+m+mi}{2}\PY{p}{]}\PY{p}{]} \PY{o}{=} \PY{n}{a}\PY{p}{[}\PY{l+m+mi}{2}\PY{p}{]}

\PY{n+nb}{print}\PY{p}{(}\PY{n}{a\PYZus{}df}\PY{p}{)}

\PY{n}{a\PYZus{}model} \PY{o}{=} \PY{n}{a\PYZus{}df}\PY{o}{.}\PY{n}{values}
\PY{n+nb}{print}\PY{p}{(}\PY{l+s+s1}{\PYZsq{}}\PY{l+s+se}{\PYZbs{}n}\PY{l+s+s1}{\PYZsq{}}\PY{p}{,} \PY{n}{a\PYZus{}model}\PY{p}{,} \PY{n}{a\PYZus{}model}\PY{o}{.}\PY{n}{shape}\PY{p}{,} \PY{l+s+s1}{\PYZsq{}}\PY{l+s+se}{\PYZbs{}n}\PY{l+s+s1}{\PYZsq{}}\PY{p}{)}
\PY{n+nb}{print}\PY{p}{(}\PY{n}{a\PYZus{}df}\PY{o}{.}\PY{n}{sum}\PY{p}{(}\PY{n}{axis}\PY{o}{=}\PY{l+m+mi}{1}\PY{p}{)}\PY{p}{)}
\end{Verbatim}
\end{tcolorbox}

    \begin{Verbatim}[commandchars=\\\{\}]
Probabilidades Iniciales de los Estados Ocultos:

deshonesto    0.333333
incierto      0.333333
honesto       0.333333
Name: Estados, dtype: float64

 1.0


Matriz de Transición De modelo:

             deshonesto  incierto      honesto
deshonesto     0.209834  0.580331     0.209834
incierto    2.65587e-05  0.999947  2.65587e-05
honesto        0.209834  0.580331     0.209834

 [[0.20983444405145937 0.580331111897081 0.20983444405145937]
 [2.655869325990881e-05 0.9999468826134802 2.655869325990881e-05]
 [0.20983444405145937 0.580331111897081 0.20983444405145937]] (3, 3)

deshonesto    1.0
incierto      1.0
honesto       1.0
dtype: float64
\end{Verbatim}

    \begin{tcolorbox}[breakable, size=fbox, boxrule=1pt, pad at break*=1mm,colback=cellbackground, colframe=cellborder]
\prompt{In}{incolor}{6}{\hspace{4pt}}
\begin{Verbatim}[commandchars=\\\{\}]
\PY{c+c1}{\PYZsh{}Matriz de probabilidad de emisión u observación.}
\PY{c+c1}{\PYZsh{}b = probabilidad de observación dada el estado.}
\PY{c+c1}{\PYZsh{}La matriz es el tamaño (M x O) donde M es el número de estados}
\PY{c+c1}{\PYZsh{}y O es el número de diferentes observaciones posibles.}

\PY{n+nb}{print}\PY{p}{(}\PY{l+s+s1}{\PYZsq{}}\PY{l+s+se}{\PYZbs{}n}\PY{l+s+s1}{ Matriz de probabilidad de Emisión u observación del entrenamiento: }\PY{l+s+se}{\PYZbs{}n}\PY{l+s+s1}{\PYZsq{}}\PY{p}{)}
\PY{n}{states} \PY{o}{=} \PY{p}{[}\PY{l+s+s1}{\PYZsq{}}\PY{l+s+s1}{CB}\PY{l+s+s1}{\PYZsq{}}\PY{p}{,} \PY{l+s+s1}{\PYZsq{}}\PY{l+s+s1}{TN}\PY{l+s+s1}{\PYZsq{}}\PY{p}{,} \PY{l+s+s1}{\PYZsq{}}\PY{l+s+s1}{CG}\PY{l+s+s1}{\PYZsq{}}\PY{p}{,} \PY{l+s+s1}{\PYZsq{}}\PY{l+s+s1}{MF}\PY{l+s+s1}{\PYZsq{}}\PY{p}{,} \PY{l+s+s1}{\PYZsq{}}\PY{l+s+s1}{VI}\PY{l+s+s1}{\PYZsq{}}\PY{p}{,} \PY{l+s+s1}{\PYZsq{}}\PY{l+s+s1}{NO}\PY{l+s+s1}{\PYZsq{}}\PY{p}{]}
\PY{n}{observable\PYZus{}states} \PY{o}{=} \PY{n}{states}

\PY{n}{b\PYZus{}df} \PY{o}{=} \PY{n}{pd}\PY{o}{.}\PY{n}{DataFrame}\PY{p}{(}\PY{n}{columns}\PY{o}{=}\PY{n}{observable\PYZus{}states}\PY{p}{,} \PY{n}{index}\PY{o}{=}\PY{n}{hidden\PYZus{}states}\PY{p}{)}
\PY{n}{b\PYZus{}df}\PY{o}{.}\PY{n}{loc}\PY{p}{[}\PY{n}{hidden\PYZus{}states}\PY{p}{[}\PY{l+m+mi}{0}\PY{p}{]}\PY{p}{]} \PY{o}{=} \PY{n}{b}\PY{p}{[}\PY{l+m+mi}{0}\PY{p}{]}
\PY{n}{b\PYZus{}df}\PY{o}{.}\PY{n}{loc}\PY{p}{[}\PY{n}{hidden\PYZus{}states}\PY{p}{[}\PY{l+m+mi}{1}\PY{p}{]}\PY{p}{]} \PY{o}{=} \PY{n}{b}\PY{p}{[}\PY{l+m+mi}{1}\PY{p}{]}
\PY{n}{b\PYZus{}df}\PY{o}{.}\PY{n}{loc}\PY{p}{[}\PY{n}{hidden\PYZus{}states}\PY{p}{[}\PY{l+m+mi}{2}\PY{p}{]}\PY{p}{]} \PY{o}{=} \PY{n}{b}\PY{p}{[}\PY{l+m+mi}{2}\PY{p}{]}

\PY{n+nb}{print}\PY{p}{(}\PY{n}{b\PYZus{}df}\PY{p}{)}

\PY{n}{b\PYZus{}model} \PY{o}{=} \PY{n}{b\PYZus{}df}\PY{o}{.}\PY{n}{values}
\PY{n+nb}{print}\PY{p}{(}\PY{l+s+s1}{\PYZsq{}}\PY{l+s+se}{\PYZbs{}n}\PY{l+s+s1}{\PYZsq{}}\PY{p}{,} \PY{n}{b\PYZus{}model}\PY{p}{,} \PY{n}{b\PYZus{}model}\PY{o}{.}\PY{n}{shape}\PY{p}{,} \PY{l+s+s1}{\PYZsq{}}\PY{l+s+se}{\PYZbs{}n}\PY{l+s+s1}{\PYZsq{}}\PY{p}{)}
\PY{n+nb}{print}\PY{p}{(}\PY{n}{b\PYZus{}df}\PY{o}{.}\PY{n}{sum}\PY{p}{(}\PY{n}{axis}\PY{o}{=}\PY{l+m+mi}{1}\PY{p}{)}\PY{p}{)}
\PY{n+nb}{print}\PY{p}{(}\PY{l+s+s1}{\PYZsq{}}\PY{l+s+se}{\PYZbs{}n}\PY{l+s+s1}{\PYZsq{}}\PY{p}{)}
\end{Verbatim}
\end{tcolorbox}

    \begin{Verbatim}[commandchars=\\\{\}]

 Matriz de probabilidad de Emisión u observación del entrenamiento:

                      CB        TN           CG           MF            VI  \textbackslash{}
deshonesto  9.56372e-132         1  1.28338e-10  4.2382e-131  1.41837e-132
incierto        0.119903  0.160677     0.239806     0.119903      0.119903
honesto     9.56372e-132         1  1.28338e-10  4.2382e-131  1.41837e-132

                     NO
deshonesto  8.10545e-94
incierto       0.239806
honesto     8.10545e-94

 [[9.563723497050358e-132 0.9999999998716623 1.2833756645962422e-10
  4.23819947420096e-131 1.4183679870705015e-132 8.10545153668598e-94]
 [0.11990323086750634 0.16067738395299896 0.23980646170946943
  0.11990323086750634 0.11990323086750634 0.2398064617350127]
 [9.563723497050358e-132 0.9999999998716623 1.2833756645962422e-10
  4.23819947420096e-131 1.4183679870705015e-132 8.10545153668598e-94]] (3, 6)

deshonesto    1.0
incierto      1.0
honesto       1.0
dtype: float64


\end{Verbatim}

    Se inserta la secuencia de observaciones del entrevistado:

    \begin{tcolorbox}[breakable, size=fbox, boxrule=1pt, pad at break*=1mm,colback=cellbackground, colframe=cellborder]
\prompt{In}{incolor}{7}{\hspace{4pt}}
\begin{Verbatim}[commandchars=\\\{\}]
\PY{c+c1}{\PYZsh{}Secuencia de Observaciones.}
\PY{n}{obs\PYZus{}vi} \PY{o}{=} \PY{n}{np}\PY{o}{.}\PY{n}{array}\PY{p}{(}\PY{p}{[}\PY{l+m+mi}{0}\PY{p}{,}\PY{l+m+mi}{0}\PY{p}{,}\PY{l+m+mi}{0}\PY{p}{,}\PY{l+m+mi}{0}\PY{p}{,}\PY{l+m+mi}{0}\PY{p}{,}\PY{l+m+mi}{0}\PY{p}{,}\PY{l+m+mi}{0}\PY{p}{,}\PY{l+m+mi}{0}\PY{p}{,}\PY{l+m+mi}{0}\PY{p}{,}\PY{l+m+mi}{0}\PY{p}{,}\PY{l+m+mi}{0}\PY{p}{,}\PY{l+m+mi}{0}\PY{p}{,}\PY{l+m+mi}{0}\PY{p}{,}\PY{l+m+mi}{0}\PY{p}{,}\PY{l+m+mi}{0}\PY{p}{,}\PY{l+m+mi}{0}\PY{p}{,}\PY{l+m+mi}{0}\PY{p}{,}\PY{l+m+mi}{0}\PY{p}{,}\PY{l+m+mi}{1}\PY{p}{,}\PY{l+m+mi}{1}\PY{p}{,}\PY{l+m+mi}{1}\PY{p}{,}\PY{l+m+mi}{1}\PY{p}{]}\PY{p}{)}

\PY{n}{array\PYZus{}nstates} \PY{o}{=} \PY{p}{(}\PY{p}{[}\PY{p}{]}\PY{p}{)}
\PY{n}{inv\PYZus{}obs\PYZus{}map} \PY{o}{=} \PY{n+nb}{dict}\PY{p}{(}\PY{p}{(}\PY{n}{v}\PY{p}{,}\PY{n}{k}\PY{p}{)} \PY{k}{for} \PY{n}{k}\PY{p}{,} \PY{n}{v} \PY{o+ow}{in} \PY{n}{obs\PYZus{}map}\PY{o}{.}\PY{n}{items}\PY{p}{(}\PY{p}{)}\PY{p}{)}

\PY{n}{obs\PYZus{}seq} \PY{o}{=} \PY{p}{[}\PY{n}{inv\PYZus{}obs\PYZus{}map}\PY{p}{[}\PY{n}{v}\PY{p}{]} \PY{k}{for} \PY{n}{v} \PY{o+ow}{in} \PY{n+nb}{list}\PY{p}{(}\PY{n}{obs\PYZus{}vi}\PY{p}{)}\PY{p}{]}

\PY{n+nb}{print}\PY{p}{(} \PY{n}{pd}\PY{o}{.}\PY{n}{DataFrame}\PY{p}{(}\PY{n}{np}\PY{o}{.}\PY{n}{column\PYZus{}stack}\PY{p}{(}\PY{p}{[}\PY{n}{obs\PYZus{}vi}\PY{p}{,} \PY{n}{obs\PYZus{}seq}\PY{p}{]}\PY{p}{)}\PY{p}{,} 
                \PY{n}{columns}\PY{o}{=}\PY{p}{[}\PY{l+s+s1}{\PYZsq{}}\PY{l+s+s1}{Código}\PY{l+s+s1}{\PYZsq{}}\PY{p}{,} \PY{l+s+s1}{\PYZsq{}}\PY{l+s+s1}{Secuencia}\PY{l+s+s1}{\PYZsq{}}\PY{p}{]}\PY{p}{)} \PY{p}{)}
\end{Verbatim}
\end{tcolorbox}

\break
    \begin{Verbatim}[commandchars=\\\{\}]
   Código Secuencia
0       0        CB
1       0        CB
2       0        CB
3       0        CB
4       0        CB
5       0        CB
6       0        CB
7       0        CB
8       0        CB
9       0        CB
10      0        CB
11      0        CB
12      0        CB
13      0        CB
14      0        CB
15      0        CB
16      0        CB
17      0        CB
18      1        TN
19      1        TN
20      1        TN
21      1        TN
\end{Verbatim}

    Se llama la función de Viterbi para calcular el estado oculto mas
probable:

    \begin{tcolorbox}[breakable, size=fbox, boxrule=1pt, pad at break*=1mm,colback=cellbackground, colframe=cellborder]
\prompt{In}{incolor}{8}{\hspace{4pt}}
\begin{Verbatim}[commandchars=\\\{\}]
\PY{n}{path}\PY{p}{,} \PY{n}{delta}\PY{p}{,} \PY{n}{phi} \PY{o}{=} \PY{n}{viterbi}\PY{p}{(}\PY{n}{pi}\PY{p}{,} \PY{n}{a\PYZus{}model}\PY{p}{,} \PY{n}{b\PYZus{}model}\PY{p}{,} \PY{n}{obs\PYZus{}vi}\PY{p}{)}
\PY{n+nb}{print}\PY{p}{(}\PY{l+s+s1}{\PYZsq{}}\PY{l+s+se}{\PYZbs{}n}\PY{l+s+s1}{ La mejor ruta del estado: }\PY{l+s+se}{\PYZbs{}n}\PY{l+s+s1}{\PYZsq{}}\PY{p}{,} \PY{n}{path}\PY{p}{)}
\PY{n+nb}{print}\PY{p}{(}\PY{l+s+s1}{\PYZsq{}}\PY{l+s+se}{\PYZbs{}n}\PY{l+s+s1}{ delta:}\PY{l+s+se}{\PYZbs{}n}\PY{l+s+s1}{\PYZsq{}}\PY{p}{,} \PY{n}{delta}\PY{p}{)}
\PY{n+nb}{print}\PY{p}{(}\PY{l+s+s1}{\PYZsq{}}\PY{l+s+s1}{phi:}\PY{l+s+se}{\PYZbs{}n}\PY{l+s+s1}{\PYZsq{}}\PY{p}{,} \PY{n}{phi}\PY{p}{)}
\end{Verbatim}
\end{tcolorbox}

    \begin{Verbatim}[commandchars=\\\{\}]

Inicio. Caminar hacia adelante

s=0 and t=1: phi[0, 1] = 1.0
s=1 and t=1: phi[1, 1] = 1.0
s=2 and t=1: phi[2, 1] = 1.0
s=0 and t=2: phi[0, 2] = 1.0
s=1 and t=2: phi[1, 2] = 1.0
s=2 and t=2: phi[2, 2] = 1.0
s=0 and t=3: phi[0, 3] = 1.0
s=1 and t=3: phi[1, 3] = 1.0
s=2 and t=3: phi[2, 3] = 1.0
s=0 and t=4: phi[0, 4] = 1.0
s=1 and t=4: phi[1, 4] = 1.0
s=2 and t=4: phi[2, 4] = 1.0
s=0 and t=5: phi[0, 5] = 1.0
s=1 and t=5: phi[1, 5] = 1.0
s=2 and t=5: phi[2, 5] = 1.0
s=0 and t=6: phi[0, 6] = 1.0
s=1 and t=6: phi[1, 6] = 1.0
s=2 and t=6: phi[2, 6] = 1.0
s=0 and t=7: phi[0, 7] = 1.0
s=1 and t=7: phi[1, 7] = 1.0
s=2 and t=7: phi[2, 7] = 1.0
s=0 and t=8: phi[0, 8] = 1.0
s=1 and t=8: phi[1, 8] = 1.0
s=2 and t=8: phi[2, 8] = 1.0
s=0 and t=9: phi[0, 9] = 1.0
s=1 and t=9: phi[1, 9] = 1.0
s=2 and t=9: phi[2, 9] = 1.0
s=0 and t=10: phi[0, 10] = 1.0
s=1 and t=10: phi[1, 10] = 1.0
s=2 and t=10: phi[2, 10] = 1.0
s=0 and t=11: phi[0, 11] = 1.0
s=1 and t=11: phi[1, 11] = 1.0
s=2 and t=11: phi[2, 11] = 1.0
s=0 and t=12: phi[0, 12] = 1.0
s=1 and t=12: phi[1, 12] = 1.0
s=2 and t=12: phi[2, 12] = 1.0
s=0 and t=13: phi[0, 13] = 1.0
s=1 and t=13: phi[1, 13] = 1.0
s=2 and t=13: phi[2, 13] = 1.0
s=0 and t=14: phi[0, 14] = 1.0
s=1 and t=14: phi[1, 14] = 1.0
s=2 and t=14: phi[2, 14] = 1.0
s=0 and t=15: phi[0, 15] = 1.0
s=1 and t=15: phi[1, 15] = 1.0
s=2 and t=15: phi[2, 15] = 1.0
s=0 and t=16: phi[0, 16] = 1.0
s=1 and t=16: phi[1, 16] = 1.0
s=2 and t=16: phi[2, 16] = 1.0
s=0 and t=17: phi[0, 17] = 1.0
s=1 and t=17: phi[1, 17] = 1.0
s=2 and t=17: phi[2, 17] = 1.0
s=0 and t=18: phi[0, 18] = 1.0
s=1 and t=18: phi[1, 18] = 1.0
s=2 and t=18: phi[2, 18] = 1.0
s=0 and t=19: phi[0, 19] = 0.0
s=1 and t=19: phi[1, 19] = 1.0
s=2 and t=19: phi[2, 19] = 0.0
s=0 and t=20: phi[0, 20] = 0.0
s=1 and t=20: phi[1, 20] = 1.0
s=2 and t=20: phi[2, 20] = 0.0
s=0 and t=21: phi[0, 21] = 0.0
s=1 and t=21: phi[1, 21] = 1.0
s=2 and t=21: phi[2, 21] = 0.0
--------------------------------------------------
Iniciar retroceso

path[20] = 1.0
path[19] = 1.0
path[18] = 1.0
path[17] = 1.0
path[16] = 1.0
path[15] = 1.0
path[14] = 1.0
path[13] = 1.0
path[12] = 1.0
path[11] = 1.0
path[10] = 1.0
path[9] = 1.0
path[8] = 1.0
path[7] = 1.0
path[6] = 1.0
path[5] = 1.0
path[4] = 1.0
path[3] = 1.0
path[2] = 1.0
path[1] = 1.0
path[0] = 1.0

 La mejor ruta del estado:
 [1. 1. 1. 1. 1. 1. 1. 1. 1. 1. 1. 1. 1. 1. 1. 1. 1. 1. 1. 1. 1. 1.]

 delta:
 [[3.18790783e-132 1.01518068e-137 1.21716978e-138 1.45934837e-139
  1.74971290e-140 2.09785086e-141 2.51525735e-142 3.01571464e-143
  3.61574721e-144 4.33516744e-145 5.19772972e-146 6.23191483e-147
  7.47187032e-148 8.95853804e-149 1.07410060e-149 1.28781291e-150
  1.54404727e-151 1.85126422e-152 2.32086144e-022 4.86996669e-023
  1.02188675e-023 2.14427039e-024]
 [3.99677436e-002 4.79200704e-003 5.74546606e-004 6.88863351e-005
  8.25925541e-006 9.90258805e-007 1.18728923e-007 1.42352253e-008
  1.70675884e-009 2.04635029e-010 2.45350979e-011 2.94168124e-012
  3.52698350e-013 4.22874253e-014 5.07012960e-015 6.07892628e-016
  7.28844185e-017 8.73861306e-018 1.40402290e-018 2.25582744e-019
  3.62441199e-020 5.82330104e-021]
 [3.18790783e-132 1.01518068e-137 1.21716978e-138 1.45934837e-139
  1.74971290e-140 2.09785086e-141 2.51525735e-142 3.01571464e-143
  3.61574721e-144 4.33516744e-145 5.19772972e-146 6.23191483e-147
  7.47187032e-148 8.95853804e-149 1.07410060e-149 1.28781291e-150
  1.54404727e-151 1.85126422e-152 2.32086144e-022 4.86996669e-023
  1.02188675e-023 2.14427039e-024]]
phi:
 [[0. 1. 1. 1. 1. 1. 1. 1. 1. 1. 1. 1. 1. 1. 1. 1. 1. 1. 1. 0. 0. 0.]
 [0. 1. 1. 1. 1. 1. 1. 1. 1. 1. 1. 1. 1. 1. 1. 1. 1. 1. 1. 1. 1. 1.]
 [0. 1. 1. 1. 1. 1. 1. 1. 1. 1. 1. 1. 1. 1. 1. 1. 1. 1. 1. 0. 0. 0.]]
\end{Verbatim}

    \hypertarget{descripciuxf3n-de-las-pruebas}{%
\subsection{Descripción de las
pruebas:}\label{descripciuxf3n-de-las-pruebas}}

En este experimento se entrevistaron a 20 personas, a cada una de ellas
se les hizo 40 preguntas por lo que se obtuvieron 800 registros en
total. Estas respuestas permiten observar si una persona es honesta,
deshonesta o insierto con respecto a los estados observables.

Estos registros se introducen en archivos txt para pasarselos como
parámetros al algoritmo de entrenamiento Baum-Welch.

    \hypertarget{implementaciuxf3n-de-las-pruebas}{%
\subsection{Implementación de las pruebas:}\label{implementaciuxf3n-de-las-pruebas}}

    \begin{tcolorbox}[breakable, size=fbox, boxrule=1pt, pad at break*=1mm,colback=cellbackground, colframe=cellborder]
\prompt{In}{incolor}{ }{\hspace{4pt}}
\begin{Verbatim}[commandchars=\\\{\}]

\end{Verbatim}
\end{tcolorbox}


    % Add a bibliography block to the postdoc
    
    
    
    \end{document}
